\documentclass{beamer}
\usepackage{polyglossia}
\usepackage{tikz}
\setdefaultlanguage{hungarian}
\title[A víz folyása]{A víz folyása}
\subtitle{avagy a titokzatos hóember}
\author[Gardner]{M.~Gardner}
\usetheme{Berkeley}
\usecolortheme{beaver}
\setbeamerfont{title}{family=\rm}
\begin{document}
	\frame{\titlepage}
	\begin{frame}
		\frametitle{A kitűzött feladat}
		\begin{itemize}
			\item A modern játékiparban nagyon fontos a fantasztikus világok megteremtése.
			
			\item A programozók ezeknek a domborzatát általában fraktálként generálják, vagyis egy egyszerű függvény használatával
			
			\begin{itemize}
				\item Ha valakit érdekel, megnézheti a "gyémánt-kocka" algoritmust.
			\end{itemize}
			
			\item amikor ezt először hallottam, nagyon meglepődtem, mert az erózió folyamatát nagyon könnyű elképzelni
			
			\item Ezért, eldöntöttem, hogy megpróbálom modellezni az eróziót, és ezáltal generálok egy képzeletbeli tájat.
		\end{itemize}
		%Content goes here
	\end{frame}
	\begin{frame}
		\frametitle{A szimuláció}
		\begin{itemize}
			\item Mivel az erózió legfontosabb eleme a víz, eldöntöttem ennek a folyását modellezni.
			
			\item Az vizet valamibe kell önteni, hogy folyjon, ezert írtam egy egyszerű függvényt:
			\begin{equation}
				k(y)=\frac{(y-0.5)}{4}
			\end{equation}
			\begin{equation}
				f(x,y)=\frac{(\frac{x}{4}+15)}{(x+15)*(0.5*k(y)^{2})}
			\end{equation}
			%More content goes here
		\end{itemize}
	\end{frame}
	\begin{frame}
		\XeTeXpdffile func.pdf width \textwidth
	\end{frame}
	% etc
	\begin{frame}
		\begin{itemize}
			\item A térképet a memóriában egy $250 \times 250$ tömb jelképezi, aminek több jellemzője van:

			
			\begin{itemize}
				\item A földnek a magassága
			
				\item Sebesség (x-re és y-re bontva)
			
				\item A víz mélysége
			
				\item Az utóbbi kettő mennyiség egy bizonyos időmenyiségre eső változása
			
			\end{itemize}
			\item ezenkívül van még néhány függvény aminek a segítségével léptetjük az időt és kirajzolhatjuk a képernyőre egy térképet a szimuláció jelenlegi állapotáról.
			
			\item Az első ilyen kirajzolás látható a következő dián.
		\end{itemize}
	\end{frame}
	\begin{frame}
		Az első kép
		\XeTeXpdffile save_0.pdf width \textwidth
	\end{frame}
	\begin{frame}
		\begin{itemize}
			\item A víznek a folyásának szimulációját több részre bontottam:
			
			\begin{itemize}
				\item A víz sodródása
				
				\item A víz szétfolyása
				
				\item A súrlódás
			\end{itemize}
		\end{itemize}
	\end{frame}
	\begin{frame}
		\frametitle{A víz sodródása}
		\begin{itemize}
			\item Ha a víz már mozog, akkor az más víztömegeket is tud mozgatni.
			
			\item Mivel csak kis időegységre számitunk minden változást, a víznek csak egy kis része folyik át a szomszédjaiba:
		\end{itemize}
	\end{frame}
	\begin{frame}
	\begin{itemize}\item($\mathbf{A}$,$\mathbf{B}$ és $\mathbf{AB}$ egynél kissebb számok, amik azt jelentik hogy az egész cellából mennyi megy át az adott irányban szomszédos cellába)\end{itemize}
				\center{% Graphic for TeX using PGF
% Title: /home/mark/programming/home/fun/rivergen/movingcell.dia
% Creator: Dia v0.97+git
% CreationDate: Tue Jan 23 17:06:39 2018
% For: mark
% \usepackage{tikz}
% The following commands are not supported in PSTricks at present
% We define them conditionally, so when they are implemented,
% this pgf file will use them.
\ifx\du\undefined%
  \newlength{\du}
\fi
\setlength{\du}{35\unitlength}
\begin{tikzpicture}[even odd rule]
\pgftransformxscale{1.000000}
\pgftransformyscale{-1.000000}
\definecolor{dialinecolor}{rgb}{0.000000, 0.000000, 0.000000}
\pgfsetstrokecolor{dialinecolor}
\pgfsetstrokeopacity{1.000000}
\definecolor{diafillcolor}{rgb}{1.000000, 1.000000, 1.000000}
\pgfsetfillcolor{diafillcolor}
\pgfsetfillopacity{1.000000}
\pgfsetlinewidth{0.100000\du}
\pgfsetdash{}{0pt}
\pgfsetmiterjoin
\pgfsetbuttcap
{\pgfsetcornersarced{\pgfpoint{0.000000\du}{0.000000\du}}\definecolor{diafillcolor}{rgb}{1.000000, 1.000000, 1.000000}
\pgfsetfillcolor{diafillcolor}
\pgfsetfillopacity{1.000000}
\fill (4.742795\du,7.112513\du)--(4.742795\du,9.050006\du)--(6.717788\du,9.050006\du)--(6.717788\du,7.112513\du)--cycle;
}{\pgfsetcornersarced{\pgfpoint{0.000000\du}{0.000000\du}}\definecolor{dialinecolor}{rgb}{0.000000, 0.000000, 0.000000}
\pgfsetstrokecolor{dialinecolor}
\pgfsetstrokeopacity{1.000000}
\draw (4.742795\du,7.112513\du)--(4.742795\du,9.050006\du)--(6.717788\du,9.050006\du)--(6.717788\du,7.112513\du)--cycle;
}\pgfsetlinewidth{0.100000\du}
\pgfsetdash{}{0pt}
\pgfsetmiterjoin
\pgfsetbuttcap
{\pgfsetcornersarced{\pgfpoint{0.000000\du}{0.000000\du}}\definecolor{diafillcolor}{rgb}{1.000000, 0.000000, 0.000000}
\pgfsetfillcolor{diafillcolor}
\pgfsetfillopacity{0.196078}
\fill (4.174048\du,6.457516\du)--(4.174048\du,8.395009\du)--(6.111540\du,8.395009\du)--(6.111540\du,6.457516\du)--cycle;
}{\pgfsetcornersarced{\pgfpoint{0.000000\du}{0.000000\du}}\definecolor{dialinecolor}{rgb}{0.000000, 0.000000, 0.000000}
\pgfsetstrokecolor{dialinecolor}
\pgfsetstrokeopacity{1.000000}
\draw (4.174048\du,6.457516\du)--(4.174048\du,8.395009\du)--(6.111540\du,8.395009\du)--(6.111540\du,6.457516\du)--cycle;
}\pgfsetlinewidth{0.100000\du}
\pgfsetdash{}{0pt}
\pgfsetmiterjoin
\pgfsetbuttcap
{\pgfsetcornersarced{\pgfpoint{0.000000\du}{0.000000\du}}\definecolor{diafillcolor}{rgb}{1.000000, 1.000000, 1.000000}
\pgfsetfillcolor{diafillcolor}
\pgfsetfillopacity{1.000000}
\fill (6.117790\du,8.400008\du)--(6.117790\du,9.025006\du)--(6.717788\du,9.025006\du)--(6.717788\du,8.400008\du)--cycle;
}{\pgfsetcornersarced{\pgfpoint{0.000000\du}{0.000000\du}}\definecolor{dialinecolor}{rgb}{0.000000, 0.000000, 0.000000}
\pgfsetstrokecolor{dialinecolor}
\pgfsetstrokeopacity{1.000000}
\draw (6.117790\du,8.400008\du)--(6.117790\du,9.025006\du)--(6.717788\du,9.025006\du)--(6.717788\du,8.400008\du)--cycle;
}% setfont left to latex
\definecolor{dialinecolor}{rgb}{0.000000, 0.000000, 0.000000}
\pgfsetstrokecolor{dialinecolor}
\pgfsetstrokeopacity{1.000000}
\definecolor{diafillcolor}{rgb}{0.000000, 0.000000, 0.000000}
\pgfsetfillcolor{diafillcolor}
\pgfsetfillopacity{1.000000}
\node[anchor=base west,inner sep=0pt,outer sep=0pt,color=dialinecolor] at (5.730291\du,8.081260\du){};
% setfont left to latex
\definecolor{dialinecolor}{rgb}{0.000000, 0.000000, 0.000000}
\pgfsetstrokecolor{dialinecolor}
\pgfsetstrokeopacity{1.000000}
\definecolor{diafillcolor}{rgb}{0.000000, 0.000000, 0.000000}
\pgfsetfillcolor{diafillcolor}
\pgfsetfillopacity{1.000000}
\node[anchor=base west,inner sep=0pt,outer sep=0pt,color=dialinecolor] at (6.380289\du,7.893760\du){A};
% setfont left to latex
\definecolor{dialinecolor}{rgb}{0.000000, 0.000000, 0.000000}
\pgfsetstrokecolor{dialinecolor}
\pgfsetstrokeopacity{1.000000}
\definecolor{diafillcolor}{rgb}{0.000000, 0.000000, 0.000000}
\pgfsetfillcolor{diafillcolor}
\pgfsetfillopacity{1.000000}
\node[anchor=base west,inner sep=0pt,outer sep=0pt,color=dialinecolor] at (5.261543\du,8.840007\du){B};
% setfont left to latex
\definecolor{dialinecolor}{rgb}{0.000000, 0.000000, 0.000000}
\pgfsetstrokecolor{dialinecolor}
\pgfsetstrokeopacity{1.000000}
\definecolor{diafillcolor}{rgb}{0.000000, 0.000000, 0.000000}
\pgfsetfillcolor{diafillcolor}
\pgfsetfillopacity{1.000000}
\node[anchor=base west,inner sep=0pt,outer sep=0pt,color=dialinecolor] at (5.261543\du,9.051674\du){};
% setfont left to latex
\definecolor{dialinecolor}{rgb}{0.000000, 0.000000, 0.000000}
\pgfsetstrokecolor{dialinecolor}
\pgfsetstrokeopacity{1.000000}
\definecolor{diafillcolor}{rgb}{0.000000, 0.000000, 0.000000}
\pgfsetfillcolor{diafillcolor}
\pgfsetfillopacity{1.000000}
\node[anchor=base west,inner sep=0pt,outer sep=0pt,color=dialinecolor] at (6.242789\du,8.840007\du){AB};
\pgfsetlinewidth{0.060000\du}
\pgfsetdash{}{0pt}
\pgfsetbuttcap{%
\definecolor{diafillcolor}{rgb}{0.000000, 0.000000, 0.000000}
\pgfsetfillcolor{diafillcolor}
\pgfsetfillopacity{1.000000}
% was here!!!
\pgfsetarrowsend{to}
\definecolor{dialinecolor}{rgb}{0.000000, 0.000000, 0.000000}
\pgfsetstrokecolor{dialinecolor}
\pgfsetstrokeopacity{1.000000}
\draw (4.824045\du,7.070014\du)--(5.392793\du,7.725011\du);
}
\end{tikzpicture}
}
	\end{frame}
	\begin{frame}
		Így megkapjuk a következő képleteket:
		\begin{equation}
			v\,'_2=\frac{\mathbf{X}\times m_1\times v_1 + m_2\times v_2}{\mathbf{X}\times m_1 + m_2}
		\end{equation}
		\begin{equation}
			m'_2=m_2+\mathbf{X}\times m_1
		\end{equation}
		Ahol $\mathbf{X}$ az $\mathbf{A}$,$\mathbf{B}$ vagy $\mathbf{AB}$ közül lehet bármelyik
	\end{frame}
	\begin{frame}
		\begin{itemize}
			\item Ezeket az arányokat a következő egyenletek határozzák meg:
		\end{itemize}
		\begin{equation}
			\mathbf{AB} = \frac{|v_x t \times v_y t|}{l^2}
		\end{equation}
		\begin{equation}
			\mathbf{A} = \frac{|v_x|\times t}{l}-\mathbf{AB}
		\end{equation}
		\begin{equation}
			\mathbf{B} = \frac{|v_y|\times t}{l}-\mathbf{AB}
		\end{equation}
		Ahol $t$ az egy lépésnyi itőt(ami valtozhat), $l$ pedig a cella szélességét
	\end{frame}
	\end{document}
