\documentclass{beamer}
\usepackage{polyglossia}
\setdefaultlanguage{hungarian}
\title[A víz folyása]{A víz folyása}
\subtitle{avagy a titokzatos hóember}
\author[Gardner]{M.~Gardner}
\usetheme{Berkeley}
\usecolortheme{beaver}
\setbeamerfont{title}{family=\rm}
\begin{document}
	\frame{\titlepage}
	\begin{frame}
		\frametitle{A kitűzött feladat}
		\begin{itemize}
			\item A modern játékiparban nagyon fontos a fantasztikus világok megteremtése.
			\pause
			\item A programozók ezeknek a domborzatát általában fraktálként generálják, vagyis egy egyszerű függvény használatával
			\pause
			\begin{itemize}
				\item Ha valakit érdekel, megnézheti a "gyémánt-kocka" algoritmust.
			\end{itemize}
			\pause
			\item amikor ezt először hallottam, nagyon meglepődtem, mert az erózió folyamatát nagyon könnyű elképzelni
			\pause
			\item Ezért, eldöntöttem, hogy megpróbálom modellezni az eróziót, és ezáltal generálok egy képzeletbeli tájat.
		\end{itemize}
		%Content goes here
	\end{frame}
	\begin{frame}
		\frametitle{A szimuláció}
		\begin{itemize}
			\item Mivel az erózió legfontosabb eleme a víz, eldöntöttem ennek a folyását modellezni.
			\pause
			\item Az vizet valamibe kell önteni, hogy folyjon, ezert írtam egy egyszerű függvényt:
			\begin{equation}
				k(y)=\frac{(y-0.5)}{4}
			\end{equation}
			\begin{equation}
				f(x,y)=\frac{(\frac{x}{4}+15)}{(x+15)*(0.5*k(y)^{2})}
			\end{equation}
			%More content goes here
		\end{itemize}
\end{frame}
	\begin{frame}
		\XeTeXpdffile func.pdf width \textwidth
	\end{frame}
	% etc
	\begin{frame}
		\begin{itemize}
			\item A térképet a memóriában egy $250 \times 250$ tömb jelképezi, aminek több jellemzője van:

			\pause
			\begin{itemize}
				\item A földnek a magassága
			\pause
				\item Sebesség (x-re és y-re bontva)
			\pause
				\item A víz mélysége
			\pause
				\item Az utóbbi kettő mennyiség egy bizonyos időmenyiségre eső változása
			\pause
			\end{itemize}
			\item ezenkívül van még néhány függvény aminek a segítségével léptetjük az időt és kirajzolhatjuk a képernyőre egy térképet a szimuláció jelenlegi állapotáról.
			\pause
			\item Az első ilzen kirajzolás látható a következő dián.
		\end{itemize}
	\end{frame}
	\begin{frame}
		\XeTeXpdffile save_0.pdf width \textwidth
	\end{frame}
\end{document}
